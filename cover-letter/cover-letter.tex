\documentclass[stdletter,letterpaper,addrfromright,orderfromdateto,dateleft,11pt,noaddrto,sigleft]{newlfm}
\topmarginskip{-0.0in}
\bottommarginskip{-0.0in}
\leftmarginsize{1.25in}
\rightmarginsize{1.5in}
\sigskipbefore{0.2in}
\sigskipafter{0in}
\noLines
\nolines
\noHeadline
\noheadline
\signature{Barney Potter}

\namefrom{}
\addrfrom{
	\includegraphics[width=6.5cm]{figures/fhcrc_logo} \\
  Vaccines and Infectious Disease Division \\
  Fred Hutchinson Cancer Research Center \\
  1100 Fairview Ave N \\
  Seattle, WA 98109, USA}
\emailfrom{bpotter@fredhutch.org}

\greetto{Dear Editor,}
\closeline{Sincerely,}

% comments
\usepackage{color}
\usepackage{ulem}
\definecolor{purple}{rgb}{0.459,0.109,0.538}
\def\tb#1#2{\sout{#1} \textcolor{purple}{#2}}
\def\tbc#1{\textcolor{purple}{[#1]}}

\begin{document}

\begin{newlfm}
Please find attached our manuscript entitled ``Rapid spread of a reassortant A/H3N2 virus during the the 2017-2018 influenza season''.
We would be grateful if you considered it for publication in \textit{Virus Evolution}.
% one sentence description of paper
In this paper we describe a new cade of A/H3N2 influenza viruses that emerged prior to the 2017-2018 North American influenza season which was the result of a reassortment between at least two different clades.
% one sentence description of previous state-of-the-field
Previous analyses of seasonal influenza viruses often neglected to analyze the virus from a full-genome perspective, instead focusing on individual segments or mutations.
% one sentence description of why out paper is important
Our paper provides a putative explanation for why North America's 2017-2018 influenza season was unusually deadly, and provides impetus for future analyses of seasonal influenza viruses that look at whole-genome dynamics.

% brief description of paper
We observe that during the 2017-2018 North American influenza season on particular subset of the A2 clade of A/H3N2 viruses showed extremely high prevalance, despite not carrying any amino acid mutations or antigenic changes that usually indicate high clade fitness.
Insead, we found incongruence between the phylogenies of the HA and NA segments that showed evidence that a reassortment event had taken place causing viruses in the A2 subclade (which we call A2/re) to have picked up a new NA segment.
Upon further analysis, we conclude that this reassortment combined the HA and PB1 segments of a virus from the A2 clade with all six other segments from the A1b clade.
Further, we show that this reassortment event preceded massive growth in the A2/clade, and that it likely would not have been fit to grow so quickly.

% brief description of why the paper matters
This paper is highlights the necessity for more in-depth monitoring and analysis of influenza from a whole-genome perspective, as analysis rooted in HA can fail to identify changes that have large public-health effects.
Additionally, we show the imortance of carefully considering both of influenza's surface glycoproteins---not just HA---in understanding annual dynamics of influenza spread.


\end{newlfm}

\end{document}
