The 2017-2018 North American influenza season caused more hospitalizations and deaths than any year since the 2009 H1N1 pandemic.
The majority of recorded influenza infections were caused by A(H3N2) viruses, with most of the virus's North American diversity falling into the A2 clade.
Within A2, we observe a subclade which we call A2/re that rose to comprise almost 70\% of A(H3N2) viruses circulating in North America by early 2018.
Unlike most fast-growing clades, however, A2/re contains no amino acid substitutions in the hemagglutinin (HA) segment.
Moreover, HI assays did not suggest substantial antigenic differences between A2/re viruses and viruses sampled during the 2016-2017 season.
Rather, we observe that the A2/re clade was the result of a reassortment event that occurred in late 2016 or early 2017 and involved the combination of the HA and PB1 segments of an A2 virus with neuraminidase (NA) and other segments a virus from the clade A1b.
The success of this clade shows the need for antigenic analysis that targets NA in addition to HA.
Our results illustrate the potential for non-HA drivers of viral success and necessitate the need for more thorough tracking of full viral genomes to better understand the dynamics of influenza epidemics.
